
% this file is called up by thesis.tex
% content in this file will be fed into the main document

%: ----------------------- introduction file header -----------------------
\selectlanguage{greek}
\chapter{Εισαγωγή}
\ifpdf
    \graphicspath{{1_introduction/figures/PNG/}{1_introduction/figures/PDF/}{1_introduction/figures/}}
\else
    \graphicspath{{1_introduction/figures/EPS/}{1_introduction/figures/}}
\fi
Σκοπός της διδακτορικής διατριβής είναι να εμπλουτίσει και να επεκτείνει υπάρχουσες μεθόδους (και λογισμικό) βελτιστοποίησης το οποίο βασίζεται στους εξελικτικούς αλγορίθμους (ΕΑ). Στόχος του νέου λογισμικού είναι, όταν αυτό χρησιμοποιείται σε πραγματικά μεγάλης κλίμακας προβλήματα της βιομηχανίας, να μειώνεται σημαντικά ο χρόνος ολοκλήρωσης του έργου, κάνοντας τη διαδικασία σχεδιασμού-βελτιστοποίησης ελκυστική για χρήση σε βιομηχανικό περιβάλλον. Οι προτεινόμενες μέθοδοι και το προγραμματισθέν λογισμικό εφαρμόζονται σε ένα φάσμα εφαρμογών σχεδιασμού-βελτιστοποίησης στις στροβιλομηχανές, θερμικές και υδροδυναμικές, οι περισσότερες από τις οποίες είναι βιομηχανικού ενδιαφέροντος.

Σε επίπεδο λογισμικού, οι προτεινόμενες μέθοδοι υλοποιούνται εντασσόμενες στο γενικής χρήσης λογισμικό βελτιστοποίησης \english{EASY (Evolutionary Algorithm SYstem)}, της Μονάδας Παράλληλης Υπολογιστικής Ρευστοδυναμικής \& Βελτιστοποίησης του Εργαστηρίου Θερμικών Στροβιλομηχανών του ΕΜΠ (ΜΠΥΡ\&Β/ΕΘΣ), \cite{phd_Giotis,phd_Karakasis,phd_Kampolis,EASYsite}. Εκμεταλλεύονται προϋπάρχουσες δυνατότητες του λογισμικού \english{EASY}, στις οποίες συμπεριλαμβάνονται η «έξυπνη» χρήση τεχνητών νευρωνικών δικτύων (ως μεταπρoτύπων, δηλαδή προσεγγιστικών προτύπων αξιολόγησης χαμηλού κόστους, υποκατάστατων του ακριβούς αλλά και ακριβού λογισμικού υπολογιστικής ρευστοδυναμικής, ΥΡΔ – αλγόριθμος \english{MAEA: Metamodel-Assisted EA)}, σχήματα κατανεμημένης ανίχνευσης του χώρου των λύσεων μέσω ΕΑ \english{(DEA: Distributed EA)}, ιεραρχικά ή πολυεπίπεδα σχήματα βελτιστοποίησης, υβριδισμός με μεθόδους βελτιστοποίησης διαφορετικές των ΕΑ αλλά και η χρήση πολυεπεξεργασίας. Καλύπτουν δε προβλήματα μονοκριτηριακής ή πολυκριτηριακής βελτιστοποίησης, με ή χωρίς περιορισμούς.

\section{Συνεισφορά της διατριβής}
Σε σχέση με την προαναφερθείσα υποδομή, η συνεισφορά της διατριβής εντοπίζεται κυρίως στα παρακάτω τρία σημεία:

	α) Προτείνεται και πιστοποιείται μια πρωτότυπη διαδικασία σχεδιασμού στροβιλομηχανών ή συνιστωσών αυτών, η οποία βασίζεται σε ένα μικρό αριθμό διαθέσιμων αρχειοθετημένων παρόμοιων σχεδιασμών υψηλής ποιότητας. Οι τελευταίοι θεωρείται ότι αποτελούν δοκιμασμένες υπάρχουσες λύσεις σε συναφή προβλήματα τα οποία, συνήθως, αφορούν λειτουργία σε λίγο διαφορετικές συνθήκες. Πάντως, ουδόλως υπονοείται ότι οι χρησιμοποιούμενοι αρχειοθετημένοι σχεδιασμοί, για τις δικές τους συνθήκες λειτουργίας, είναι βέλτιστοι (με την αυστηρή έννοια του όρου). Η προτεινόμενη διαδικασία θα αναφέρεται ως \english{Knowledge Based Design} ή \english{KBD}. Η διαδικασία \english{KBD} επιχειρεί να δώσει απάντηση στον ενδοιασμό των μηχανικών της βιομηχανίας για το αν κάθε νέος σχεδιασμός πρέπει να ξεκινά «από το μηδέν» ή μπορεί να στη-ριχθεί στην υπάρχουσα εμπειρία. Για την εφαρμογή της μεθόδου \english{KBD}, πρέπει αρ-χικά να απομονωθεί ένα μικρό σύνολο «συναφών» σχεδιασμών του παρελθόντος. Αν λ.χ. το τρέχον πρόβλημα αφορά στο σχεδιασμό μιας στροβιλομηχανής σε συνθήκες «Α», με στόχους «Β» και περιορισμούς «Γ», το σύνολο αυτό μπορεί να αποτελείται από ένα (συνήθως μονοψήφιο) αριθμό στροβιλομηχανών που έχουν σχεδιαστεί στο παρελθόν και ήδη λειτουργούν με καλή απόδοση σε συνθήκες διαφορετικές μεν, πλησίον δε των «Α», για ίδια ή περίπου ίδια με τα «Β» κριτήρια απόδοσης («στόχους» στην ορολογία των μεθόδων βελτιστοποίησης) και ικανοποιούν περιορισμούς περισσότερο ή λιγότερο συναφείς με τους «Γ». Ο τρόπος επιλογής του συνόλου των απαραίτητων «συναφών» σχεδιασμών, οι οποίοι πλέον θα αποκαλούνται «σχεδιασμοί βάσης»,  δεν εμπίπτει στα ενδιαφέροντα της διατριβής. Επίσης, σε αυτά δεν εμπίπτει η διαδικασία ενιαίας παραμετροποίησης των σχεδιασμών βάσης σε τρόπο συμβατό με το προς υλοποίηση πρόβλημα σχεδιασμού. Για την περίπτωση που αυτή δεν υφίσταται, επισημαίνεται ότι αυτή μπορεί να υλοποιηθεί με πολλούς τρόπους. Η διατριβή προτείνει ένα νέο μαθηματικό τρόπο διατύπωσης του προβλήματος του  νέου σχεδιασμού συναρτήσει των σχεδιασμών βάσης. Ο τρόπος αυτός παρακάμπτει ουσιαστικά την παραμετροποίηση της σχεδιαζόμενης γεωμετρίας, η οποία εκ των πραγμάτων μπορεί να εισάγει εκατοντάδες βαθμούς ελευθερίας, με αποτέλεσμα να καθυστερεί η σύγκλιση της βασισμένης στους ΕΑ μεθόδου βελτιστοποίησης. Αντ’ αυτών, εισάγεται ένα μικρό πλήθος νέων μεταβλητών σχεδιασμού (θα ονομάζονται μεταβλητές βελτιστοποίησης ώστε να υπάρχει διάκριση με τις προηγούμενες) και καθορίζονται περιοχές  μεγαλύτερης σημαντικότητας, με όφελος την αισθητή μείωση του χρόνου βελτιστοποίησης. Επιπλέον σημαντικό κέρδος από τη νέα παραμε-τροποίηση αποτελεί το ότι τα όρια των νέων μεταβλητών σχεδιασμού προκύπτουν εύκολα, ουσιαστικά «αυτόματα» και χωρίς παρέμβαση του χρήστη. Η προτεινόμενη μέθοδος, αφού προγραμματίστηκε και εντάχθηκε στο λογισμικό \english{EASY}, χρησιμοποιήθηκε για το σχεδιασμό στροβιλομηχανών επιφέροντας κέρδος μιας τάξης μεγέθους σε υπολογιστικό χρόνο για την επίτευξη  παρόμοιας ποιότητας σχεδιασμών.       
               
	β)  Προτείνεται μέθοδος επίλυσης προβλημάτων βελτιστοποίησης που χαρακτηρίζονται απο μη-διαχωρίσιμες (ως προς τις μεταβλητές σχεδιασμού) συναρτήσεις κόστους ή καταλληλότητας, με μεγάλο αριθμό μεταβλητών σχεδιασμού. Αυτά θα ονομάζονται «κακώς-τοποθετημένα» προβλήματα βελτιστοποίησης και η αντιμετώπιση τους μέσω ΕΑ οδηγεί, σχεδόν πάντα, σε πολύ χρονοβόρους υπολογισμούς. Η προτεινόμενη μέθοδος υλοποιείται επεμβαίνοντας κατάλληλα στον τρόπο που εφαρμόζονται οι τελεστές εξέλιξης.  Αυτό επιτυγχάνεται μέσω της δυναμικά (δηλαδή, σε κάθε γενιά) ανανεούμενης επαναδιατύπωσης του προβλήματος βελτιστοποίησης, ώστε ο ΕΑ να χειρίζεται, κατά το δυνατό, προβλήματα των οποίων η συνάρτηση-στόχος να είναι διαχωρίσιμη ως προς τις μεταβλητές σχεδιασμού. Ο λόγος που μια τέτοια αντιμετώπιση επιφέρει μείωση στο χρόνο επίλυσης ενός προβλήματος βελτιστοποίησης είναι ότι οι ΕΑ, εκ φύσεως, κερδίζουν σημαντικά σε ταχύτητα όταν χειρίζονται προβλήματα ελαχιστοποίησης διαχωρίσιμων συναρτήσεων-στόχων. Αυτό οφείλεται στο ότι κάθε πρόβλημα ελαχιστοποίησης μιας διαχωρίσιμης συνάρτησης (Ν σε πλήθος μεταβλητών) μπορεί ιδεατά να αντιμετωπιστεί ως Ν διακριτά προβλήματα ελαχιστοποίησης κατάλληλων συναρτήσεων μιας μεταβλητής, με συνολικά μικρότερο υπολογιστικό κόστος.  Αυτό εκμεταλλεύεται η προτεινόμενη μέθοδος. Προαπαίτηση για την υλοποίησή της είναι η αυτόματη μετατροπή του πραγματικού προβλήματος σε πρόβλημα διαχωρίσιμων μεταβλητών, στο βαθμό που αυτό είναι εφικτό. Αυτό πραγματοποιείται με τη μέθοδο Ανάλυσης σε Κύριες Συνιστώσες \cite{Haykin,Jolliffe_2002} (ΑσΚΣ, \english{Principal Component Analysis, PCA}). Σε προβλήματα πολυκριτηριακής βελτιστοποίησης, η ΑσΚΣ εφαρμόζεται στο σύνολο των επιλέκτων κάθε γενιάς και, ουσιαστικά, πραγματοποιεί κατάλληλη «στροφή» του χώρου σχεδιασμού. Η «στροφή» αυτή απαιτεί την επίλυση ενός προβλήματος ιδιοτιμών με ασήμαντο υπολογιστικό κόστος. Αποδεικνύεται ότι ο προκύπτων χώρος, ίδιας διάστασης με το χώρο σχεδιασμού, ο οποίος καθορίζεται με άξονες τα ιδιοδιανύσματα (τις λεγόμενες «κύριες συνιστώσες») που προκύπτουν από την ΑσΚΣ, έχει τις προαναφερθείσες ιδιότητες διαχωρισιμότητας. Οι τελεστές εξέλιξης (διασταύρωση, μετάλλαξη, κλπ) εφαρμόζονται στις νέες μεταβλητές σχεδιασμού και, έτσι, προκύπτουν νέοι απόγονοι, οι οποίοι τελικά επαναφέρονται (με «αντίθετη στροφή») στον αρχικό-πραγματικό χώρο σχεδιασμού. Η προτεινόμενη μέθοδος θα αναφέρεται ως \english{EA(PCA)} ή \english{MAEA(PCA)}, τονίζοντας έτσι ότι η χρήση \english{PCA} ή ΑσΚΣ αφορά στους ΕΑ και όχι στα μεταπρότυπα. Η εναλλακτική επιλογή παρουσιάζεται στο (γ) και προγραμματίστηκε συμπληρωματικά στον \english{EASY}. Αρχικά, πιστοποιήθηκε σε μαθηματικές συναρτήσεις και ψευδο-μηχανολογικά προβλήματα χαμηλού κόστους της βιβλιογραφίας. Στη συνέχεια, χρησιμοποιήθηκε για το σχεδιασμό του δρομέα ενός υδροστροβίλου  \english{Hydromatrix\circledR} με κέρδος την αναπαραγωγή παρόμοιας ποιότητας σχεδιασμών στο μισό περίπου χρόνο (μισός αριθμός κλήσεων του λογισμικού ΥΡΔ, το οποίο χρησιμοποιείται για την αξιολόγηση των υποψήφιων λύσεων) σε σχέση με τον κλασικό EA.  
 
	γ) Ως προς τον ΕΑ που υποβοηθείται από μεταπρότυπα (\english{MAEA}, στη λογική της προσεγγιστικής προ-αξιολόγησης – ΠΠΑ – των ατόμων κάθε γενιάς με τεχνητά νευρωνικά δίκτυα \cite{LTT_2_018,LTT_2_020,LTT_2_029}) προτείνεται και πιστοποιείται μέθοδος η οποία αντιμετωπίζει με επιτυχία ένα σημαντικό πρόβλημα το οποίο ανακύπτει όταν αυτές χρησιμοποιούνται σε προβλήματα μεγάλης διάστασης. Από τη μέχρι σήμερα εμπειρία από τη χρήση \english{MAEA}, είναι γνωστό ότι το κέρδος (συγκριτικά με τους κλασικούς ΕΑ) μειώνεται όταν η διάσταση του χώρου σχεδιασμού αυξάνει σημαντικά. Αυτή η συμπεριφορά οφείλεται (\english{i}) στο ότι η έναρξη χρήσης των τεχνητών νευρωνικών δικτύων καθυστερεί, αναμένοντας την καταγραφή επαρκούς πλήθους ήδη αξιολογημένων υποψηφίων λύσεων στη βάση δεδομένων από την οποία αντλούνται τα δείγματα εκπαίδευσης των μεταπροτύπων και (\english{ii}) στο ότι η αξιοπιστία των νευρωνικών δικτύων φθίνει καθώς αυξάνει ο αριθμός εισόδων σε αυτά, άρα το πλήθος μεταβλητών σχεδιασμού. Η προτεινόμενη αντιμετώπιση αυτού του προβλήματος βασίζεται στην ελεγχόμενη μείωση των εισόδων (κρατώντας, ουσιαστικά, τις περισσότερο αντι-προσωπευτικές) του νευρωνικού δικτύου. Εδώ, χρησιμοποιούνται δίκτυα ακτινικών συναρτήσεων βάσης, \english{Radial Basis Function network, RBF}\cite{Haykin}) που χρησιμοποι-ούνται  ως μεταπρότυπα.  Μέσω ΑσΚΣ, εφαρμοζόμενης στο σύνολο των δυναμικά ανανεούμενων επιλέκτων, πραγματοποιείται εκ νέου στροφή/ευθυγράμμιση του χώρου σχεδιασμού με τις κύριες συνιστώσες λαμβάνοντας υπόψη τη σημαντικότητα κάθε μεταβλητής. Η τελευταία είναι αντιστρόφως ανάλογη της τιμής της σχετικής ιδιοτιμής που προέκυψε από την ΑσΚΣ. Εδώ, επιπλέον στοιχείο είναι ότι στο «στραμμένο» χώρο σχεδιασμού, γίνεται αποκοπή και κρατείται μικρός αριθμός των πλέον σημαντικών «στραμμένων» μεταβλητών σχεδιασμού. Με τις τελευταίες και μόνο αυτές, εκπαιδεύεται το δίκτυο \english{RBF}. Η τεχνική αυτή θα αναφέρεται ως \english{M(PCA)AEA}, υποδηλώνοντας τη χρήση της ΑσΚΣ κατά την εκπαίδευση των μεταπροτύπων. Η \english{M(PCA)AEA} οδηγεί σε περαιτέρω μείωση του χρόνου βελτιστοποίησης αφού τα μεταπρότυπα παρέχουν προβλέψεις υψηλότερης αξιοπιστίας αλλά, επίσης, μπορούν να  ξεκινήσουν να χρησιμοποιούνται νωρίτερα κατά τη διαδικασία βελτιστοποίησης. Η \english{M(PCA)AEA} χρησιμοποιήθηκε, με ή χωρίς την επικουρική χρήση της μεθόδου (α) και σε συνδυασμό με τη μέθοδο (β), στο σχεδιασμό-βελτιστοποίηση 2Δ και 3Δ πτερύγωσης συμπιεστή, με κέρδος τη μείωση του χρόνου βελτιστοποίησης στο 1/3 του χρόνου του προϋπάρχοντος  \english{MAEA}. Η πλέον ενισχυμένη παραλλαγή θα αναφέρεται ως \english{M(PCA)AEA(PCA)}.
	       
\section{Μελετούμενες Εφαρμογές}
Οι τεχνικές που αναπτύχθηκαν στην παρούσα διδακτορική εργασία χρησιμοποιήθηκαν στο σχεδιασμό-βελτιστοποίηση ενός καινοτόμου τύπου υδροστροβίλου, του  \english{Hydromatrix\circledR}, ιδανικού για τοποθεσίες μικρού ύψους και μικρών παροχών. Οι υδροστρόβιλοι \english{Hydromatrix\circledR}, πατενταρισμένοι από την εταιρία \english{Andritz-Hydro}, έχουν υψηλότερη αποδοτικότητα από άλλους τύπους υδροστροβίλων χαμηλού ύψους, χαμηλό κόστος εγκατάστασης και περιβαλλοντικά και οικονομικά οφέλη. Κάνοντας χρήση των τεχνικών που αναπτύχθηκαν στην παρούσα διατριβή, επιτεύχθηκε μείωση του συνολικού χρόνου σχεδιασμού τους άνω του 50\%.  Η μείωση του συνολικού χρόνου σχεδιασμού και του κόστους μετατρέπουν σε οικονομικά επικερδή τη χρήση \english{Hydromatrix\circledR} σε ακόμη μικρότερα ύψη και παροχές μικρότερες. 

Σε συνεργασία με την εταιρία \english{Andritz-Hydro}, η οποία χρηματοδότησε τμήμα της  παρούσας διατριβής, οι αναπτυχθείσες τεχνικές εφαρμόστηκαν και σε μια σειρά άλλων σχεδιασμών υδροδυναμικών μηχανών ή συνιστωσών τους. Περισσότερο αναλυτικά, με βάση τις τεχνικές αυτές οργανώθηκαν διαδικασίες σχεδιασμού δρομέων όλων των κλασικών τύπων υδροδυναμικών στροβιλομηχανών αντίδρασης τόσο αξονικής όσο και μικτής ροής (\english{Francis, Kaplan, Bulb, Pump} και \english{Pump-Turbines}), με στόχο τόσο την αύξηση της απόδοσής τους όσο και τη βέλτιστη συνεργασία τους με τα υπόλοιπα μέρη της εγκατάστασης. Επίσης, σχεδιάστηκαν σταθερές συνιστώσες υδροδυναμικών μηχανών όπως αγωγοί απαγωγής (\english{draft tubes}), με στόχο τη μέγιστη ανάκτηση πίεσης με τις ελάχιστες απώλειες, αλλά και τμήματα αγωγών εισόδου υδροστροβίλων δράσης (διανομείς-\english{distributors}), με στόχο την ελαχιστοποίηση του κόστους κατασκευής και των απωλειών και με παράλληλη αύξηση της ποιότητας της δέσμης ρευστού μετά το ακροφύσιο. Επιλεγμένο τμήμα των παραπάνω σχεδιασμών παρουσιάζεται στο πλήρες κείμενο της διατριβής. 
		
Το ερευνητικό  έργο με τίτλο «\english{HYDROACTION – Development and Laboratory Testing of Improved Action and Matrix Hydro Turbines Designed by Advanced Analysis and Optimization Tools» (FP7: Project Number 211983)}, το οποίο χρηματοδότησε η Ευρωπαϊκή Ένωση, υποστήριξε το υπόλοιπο τμήμα της διατριβής. Το ΕΜΠ και η \english{Andritz-Hydro} υπήρξαν εταίροι στο έργο αυτό. Οι βιομηχανικής κλίμακας υπολογισμοί στις προαναφερθείσες εφαρμογές  στροβιλομηχανών πραγματοποιήθηκαν, εκ των πραγμάτων, σε πολυεπεξεργαστικά συστήματα. Ως τέτοια χρησιμοποιήθηκαν τα πολυεπεξεργαστικά συστήματα της ΜΠΥΡ\&Β/ΕΘΣ (ΕΜΠ, Αθήνα) και της \english{Andritz-Hydro (\greek{στις εγκαταστάσεις της εταιρίας στις πόλεις} Linz, Graz \greek{και} Vevey)}, ενίοτε και συνεργατικά μέσω τεχνικών \english{Grid Computing} \cite{phd_Liakopoulos}, τις οποίες υποστηρίζει το λογισμικό \english{EASY}. 

\section{Βιβλιογραφική επισκόπηση}
Η βιβλιογραφική επισκόπηση ως προς τη χρήση μεθόδων σχεδιασμού και βελ-τιστοποίησης στις θερμικές αλλά και υδροδυναμικές μηχανές αναδεικνύει ότι τέτοιες εφαρμογές υπάρχουν πολλές. Στις περισσότερες περιπτώσεις, οι σχεδιαστές επωφελούνται της «με ελάχιστο κόπο» υλοποίησης της βελτιστοποίησης μέσω ΕΑ, όταν είναι διαθέσιμο το λογισμικό ΥΡΔ και αυτό της παραμετροποίησης της γεωμετρίας τους. Στις περισσότερες περιπτώσεις, δεν δίνεται έμφαση στο υπολογιστικό κόστος αλλά, κυρίως, στις νέες βέλτιστες λύσεις που αναδεικνύουν οι ΕΑ. Την τελευταί-α δεκαετία, όλα τα γνωστά επιστημονικά συνέδρια στην περιοχή των στροβιλομηχανών έχουν ιδιαίτερες συνεδρίες σε θέματα βελτιστοποίησης, με τους ΕΑ να διεκδικούν συνήθως σημαντικό μερίδιο στις εκεί παρουσιαζόμενες εργασίες. Εφαρμογές στις θερμικές στροβιλομηχανές συναντώνται σε αριθμό δημοσιεύσεων της ΜΠΥΡ\&Β/ΕΘΣ \cite{LTT_2_018,LTT_2_020,LTT_2_023,LTT_2_026,LTT_2_031,LTT_2_040, LTT_2_045}.
Με την εφαρμογή ΕΑ στις υδροδυναμικές μηχανές ασχολείται και το Εργαστήριο Υδροδυναμικών Μηχανών του ΕΜΠ. Το έργο του εκτείνεται από το σχεδιασμό βέλτιστων υδροδυναμικών μηχανών και συνιστωσών αυτών \cite{Anagno4} μέχρι τη βελτιστοποίηση ολοκληρωμένων υδροηλεκτρικών σταθμών και σταθμών αποθήκευσης ενέργειας σε συνεργασία με άλλες μορφές ανανεώσιμων πηγών ενέργειας \cite{Anagno3,Anagno5}.             


 
\section{Δομή της Διατριβής} % section headings are printed smaller than chapter names

Το Κεφάλαιο $2$ παρουσιάζει, σε συντομία, τους ΕΑ στους οποίους βασίστηκε αλλά και επέκτεινε η παρούσα διατριβή με όσα προαναφέρθηκαν. Παρουσιάζεται ο γενικευμένος ΕΑ και η σύζευξή του με τα μεταπρότυπα.

Το Κεφάλαιο $3$ παρουσιάζει την προτεινόμενη μέθοδο σχεδιασμού στη βάση αρχειοθετημένης γνώσης (τεχνική \english{KBD}) και αποτιμά την επιτάχυνση της διαδικασίας σχεδιασμού-βελτιστοποίησης μέσω της εκμετάλλευσης αρχειοθετημένων σχεδιασμών, που επιταχύνεται μέσω του υβριδισμού των ΕΑ με συστήματα που βασίζο-νται στη γνώση \english{(Knowledge Based Systems)}.  


Το Κεφάλαιο $4$ ασχολείται με την έννοια των «κακώς-τοποθετημένων» (\english{ill-posed}) προβλημάτων βελτιστοποίησης και τη βέλτιστη αντιμετώπιση τους μέσω ΕΑ.
\newline
Ορίζονται ισότροπες και διαχωρίσιμες συναρτήσεις-στόχοι και προσδιορίζονται τα λεγόμενα «κακώς τοποθετημένα» προβλήματα βελτιστοποίησης και τρόποι αντιμετώπισής τους.  Εξετάζεται η επίδρασή τους στην απόδοση των ΕΑ και προτείνεται η χρήση της ΑσΚΣ για τον εντοπισμό τους καθώς και η εισαγωγή νέων τελεστών εξέλιξης για την αντιμετώπιση τους.  Ακόμη, προτείνεται η χρήση των ιδιοτιμών που προκύπτουν από την ΑσΚΣ για τη βελτίωση της απόδοσης των τοπικών μεταπροτύπων που χρησιμοποιούνται στους ΜΑΕΑ. Το κέρδος από τη χρήση της προτεινόμενης μεθόδου ποσοτικοποιείται κατά το σχεδιασμό-βελτιστοποίηση μιας 2Δ πτερύγωσης συμπιεστή.  

Στο Κεφάλαιο $5$ πιστοποιούνται οι προτεινόμενες στα Κεφάλαια 3 και 4 μέθοδοι στο σχεδιασμό-βελτιστοποίηση υδροδυναμικών μηχανών.  Αρχικά παρουσιάζονται τα «υπολογιστικά εργαλεία»,  δηλαδή η παραμετροποίηση των προς σχεδιασμό μορφών, η διαδικασία γένεσης υπολογιστικού πλέγματος και ο επιλύτης των εξισώσεων της ροής.  Στη συνέχεια, παρουσιάζονται οι μετρικές ποιότητας που χαρακτηρίζουν την απόδοση κάθε υποψήφιας λύσης. Αυτές, καθεμιά ξεχωριστά ή σε γραμμικούς συνδυασμούς με κατάλληλους συντελεστές βαρύτητας, αποτελούν τους στόχους των προβλημάτων βελτιστοποίησης.    Η προτεινόμενη μέθοδος \english{KBD} χρησιμοποιείται στο σχεδιασμό-βελτιστοποίηση ενός υδροστροβίλου τύπου \english{Francis} και η \english{MAEA(PCA)} στο σχεδιασμό ενός υδροστροβίλου τύπου \english{Hydromatrix\circledR}.   


Το Κεφάλαιο $6$ αναφέρεται στη βελτιστοποίηση μορφής των πτερυγίων της πτερύγωσης συμπιεστή η οποία είναι εγκατεστημένη στο ΕΘΣ/ΕΜΠ. Για τη διαδικασία βελτιστοποίησης εφαρμόζεται η μέθοδος \english{M(PCA)AEA(PCA)}. Πρόκειται για εφαρμογή στην περιοχή της αεροδυναμικής υψηλών ταχυτήτων, σε αντίθεση με τις εφαρμογές ασυμπίεστων ροών του Κεφαλαίου 5. 

Τέλος, στο Κεφάλαιο $7^o$ παρουσιάζονται τα συμπεράσματα που προέκυψαν από την ανάπτυξη μεθόδων βελτιστοποίησης και τις εφαρμογές που πραγματοποιήθηκαν.

Τα κεφάλαια, όπως αριθμούνται στην εκτενή περίληψη στην Ελληνική γλώσσα, αντιστοιχούν σε αυτά του πλήρους κειμένου της διατριβής στην Αγγλική γλώσσα. Γενικά, όμως, ενδέχεται οι ενότητες των κεφαλαίων να αριθμούνται διαφορετικά. 
%: ----------------------- HELP: latex document organisation
% the commands below help you to subdiv, γενide and organise your thesis
%    \chapter{}       = level 1, top level
%    \section{}       = level 2
%    \subsection{}    = level 3
%    \subsubsection{} = level 4
% note that everything after the percentage sign is hidden from output

%: ----------------------- HELP: references
% References can be links to figures, tables, sections, or references.
% For figures, tables, and text you define the target of the link with \label{XYZ}. Then you call cross-link with the command \ref{XYZ}, as above
% Citations are bound in a very similar way with \cite{XYZ}. You store your references in a BibTex file with a programme like BibDesk.


%%%%%%%% template for figures
%see fig \ref{A common glucose polymers}
%\figuremacro{EAvsPCA_zdt3}{A common glucose polymers}{The figure shows starch granules in potato cells, %taken from \href{http://molecularexpressions.com/micro/gallery/burgersnfries/burgersnfries4.html}{Molecular %Expressions}.}

%%: ----------------------- HELP: adding figures with macros
%% This template provides a very convenient way to add figures with minimal code.
%% \figuremacro{1}{2}{3}{4} calls up a series of commands formating your image.
%% 1 = name of the file without extension; PNG, JPEG is ok; GIF doesn't work
%% 2 = title of the figure AND the name of the label for cross-linking
%% 3 = caption text for the figure

%%: ----------------------- HELP: www links
%% You can also see above how, www links are placed
%% \href{http://www.something.net}{link text}

%\figuremacroW{EAvsPCA_zdt3}{Title}{Caption}{0.8}
%% variation of the above macro with a width setting
%% \figuremacroW{1}{2}{3}{4}
%% 1-3 as above
%% There you go. You already know the most important things.



