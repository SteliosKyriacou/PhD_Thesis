\cleardoublepage %put on right page
% Thesis Abstract -----------------------------------------------------
\begin{abstractslong}    %uncommenting this line, gives a different abstract heading
%\begin{abstracts}        %this creates the heading for the abstract page

The scope of this PhD thesis is to propose a set of improvements to existing shape design-optimization methods in fluid dynamics based on Evolutionary Algorithms (EAs) and demonstrate their efficiency in real-world applications. Though the proposed method and the developed EA-based software are both generic, this thesis focuses on applications in the fields of hydraulic and thermal turbomachines. With the proposed algorithmic variants, the optimization turn-around time is noticeably reduced with respect to that of conventional (reference, background) methods. Though the latter are computationally expensive, with the proposed add-ons, they become affordable even for large-scale industrial applications. The background design-optimization methods are based on EAs enhanced by the use of artificial neural networks, acting as surrogate evaluation models or metamodels. The design process is coupled with the necessary Computational Fluid Dynamic (CFD) software. Parallelization, in the form of concurrent evaluations of candidate solutions within each generation of the EA, is absolutely necessary, given the high CPU cost per CFD-based evaluation. All computations were performed on the multi-processor platform of the Parallel CFD \& Optimization Unit (PCOpt) of the National Technical University of Athens (NTUA). Using the proposed optimization methods several turbomachinery design optimization problems are solved; these include the design-optimization of traditional hydraulic turbines (such as Francis turbine) and a new/innovative variation of bulb turbines, the so-called ``Hydromatrix$\circledR$", suitable for low head hydropower sites. These computations were performed in close collaboration with a major hydraulic turbine manufacturer (Andritz Hydro).  Furthermore, the compressor cascade installed at the Lab of Thermal Turbomachines of NTUA (LTT/NTUA) is optimized. This thesis presents also the application of the proposed methods and tools on a number of mathematical optimization problems, since these allow a great number of test runs to be performed at negligible CPU cost.

The most important contributions of this thesis are listed below:
	
a)	 A new design method, which fully exploits archived designs with good performance in ``similar" conditions, is proposed. In this method, EAs (either in their conventional form or in enhanced variants assisted by metamodels) solve a reformulated optimization problem, instead of the high-dimensional real one. The new unknowns are used to non-linearly weight the archived designs so as to create candidate solutions to the problem in hand; it is a great advantage of the proposed method that, by doing so, the EA avoids handling the, otherwise, great number of variables parameterizing the complex shape to be designed. Gains from the reduction of the number of design variables are evident. An extra gain in CPU cost arises from the statistical analysis of the archived designs that helps identifying the most important design space regions, where greater probabilities of accommodating new offspring are given. The proposed method has the additional advantage of allowing the automatic definition of the new variables' upper and lower bounds.
	            
b)	 The use of principal component analysis (PCA) as a means to reveal the topological characteristics of the current generation elite set is proposed. The principal directions on the design space, as computed via PCA in each generation, are used for the rotation of the design variable coordinate system before the application of the evolution operators. This rotation suffices to transform an ill-posed optimization problem into a better-posed one. Through the proposed PCA-driven evolution operators, offspring generations of higher quality are created and, this certainly reduces the number of generations required to get the optimal solution(s). Note that this is done without truncating the design variables, as a few other method are doing.           

c)	 The same PCA technique, along with the information about the design variable importance is used to enchance the inexact pre-evaluation (IPE) phase in metamodel-assisted EAs (MAEAs). This is, herein, applied in MAEAs incorporating radial basis function (RBF) networks as metamodels, in either single- or multi-objective optimization problems. The proposed method aims at obtaining more relevant predictions of the objective function values by the online trained metamodels. To this end, the PCA-driven rotation of the design variables followed by an appropriate truncation of the less-significant among them allow the use of metamodels trained on patterns of smaller dimension. Consequently, the training can rely on a smaller number of patterns and evaluations on the RBF network are much better. The EA based search is, thus, better driven and this leads to lower CPU costs.

The gain from the use of the proposed methods, either separately or in combination, is quantified in a few mathematical test cases and, then, some real-world applications. These were solved twice, using both the proposed and the background optimization methods. In more detail, the gain from the use of KBD is demonstrated by solving the design problem of a 3D Francis hydraulic-turbine  and that of a 2D compressor cascade. The use of an EA with PCA-driven evolution operators is shown in the design-optimization of a ``Hydromatrix$\circledR$". The use of PCA to enhance the performance of metamodels in MAEAs is demonstrated in the design of a 2D compressor cascade and the optimization of the blade shape of a peripheral compressor cascade installed and measured at LTT/NTUA. 

%\end{abstracts}
\end{abstractslong}

% ---------------------------------------------------------------------- 
