
% Thesis Abstract -----------------------------------------------------


\begin{abstractslongGR}    %uncommenting this line, gives a different abstract heading
\selectlanguage{greek}
%\begin{abstracts}        %this creates the heading for the abstract page
Σκοπός της διδακτορικής διατριβής είναι να εμπλουτίσει/επεκτείνει υπάρχουσες μεθόδους (και λογισμικό) βελτιστοποίησης το οποίο βασίζεται στους εξελικτικούς αλγορίθμους (ΕΑ) ώστε, όταν χρησιμοποιείται σε πραγματικά, μεγάλης κλίμακας προβλήματα, να μειώνεται σημαντικά ο χρόνος αναμονής του σχεδιαστή, κάνοντας τη διαδικασία σχεδιασμού-βελτιστοποίησης «αποδεκτή» σε βιομηχανικό περιβάλλον. Οι προτεινόμενες μέθοδοι και το προγραμματιζόμενο λογισμικό εφαρμόζονται σε ευρύ φάσμα εφαρμογών σχεδιασμού-βελτιστοποίησης στις στροβιλομηχανές, θερμικές και υδροδυναμικές, βιομηχανικού ενδιαφέροντος.

Σε επίπεδο λογισμικού, οι προτεινόμενες μέθοδοι υλοποιούνται εντασσόμενες στο γενικής χρήσης λογισμικό βελτιστοποίησης \english{EASY (Evolutionary Algorithm System)} της ΜΠΥΡ\&Β/ΕΘΣ. Εκμεταλλεύονται τις προϋπάρχουσες δυνατότητες του λογισμικού \english{EASY}, στις οποίες συμπεριλαμβάνονται η «έξυπνη» χρήση τεχνητών νευρωνικών δικτύων (ως μεταπρότυπα, δηλαδή χαμηλού κόστους προσεγγιστικά πρότυπα αξιολόγησης, υποκατάστατα του λογισμικού υπολογιστικής ρευστοδυναμικής – αλγόριθμος \english{MAEA: Metamodel-assisted EA)}, σχήματα κατανεμημένης ανίχνευσης του χώρου των λύσεων \english{(DEA: Distributed EA)}, ιεραρχικά ή πολυεπίπεδα σχήματα βελτιστοποίησης, ο υβριδισμός με μεθόδους βελτιστοποίησης διαφορετικές των ΕΑ και, φυσικά, η χρήση πολυεπεξεργασίας. Καλύπτουν προβλήματα μονοκριτηριακής ή πολυκριτηριακής βελτιστοποίησης, με ή χωρίς περιορισμούς.

Η συνεισφορά της διατριβής, σε σχέση με την προαναφερθείσα υποδομή, εντοπίζεται κυρίως στα παρακάτω τρία σημεία:
	α) Προτείνεται και πιστοποιείται μια πρωτότυπη διαδικασία σχεδιασμού νέων προϊόντων (εδώ στροβιλομηχανών) που βασίζεται σε ένα μικρό αριθμό διαθέσιμων αρχειοθετημένων σχεδιασμών, οι οποίοι θεωρούνται βέλτιστοι ή, έστω, αποδεκτοί, για λειτουργία σε διαφορετικές συνθήκες. Η προτεινόμενη διαδικασία αποτελεί απάντηση στον ενδοιασμό των μηχανικών της βιομηχανίας για το αν κάθε νέος σχεδιασμός προϊόντος πρέπει να ξεκινά «από το μηδέν» ή μπορεί να στηριχθεί στην υπάρχουσα εμπειρία. Για την εφαρμογή της προτεινόμενης μεθόδου, πρέπει αρχικά να απομονωθεί ένα μικρό σύνολο «σχετικών» σχεδιασμών του παρελθόντος. Αν λ.χ. το τρέχον πρόβλημα αφορά το σχεδιασμό μιας στροβιλομηχανής σε συνθήκες «Α», με στόχους «Β» και περιορισμούς «Γ», το σύνολο αυτό μπορεί να αποτελείται από ένα μονοψήφιο αριθμό στροβιλομηχανών που έχουν σχεδιαστεί στο παρελθόν για λειτουργία σε συνθήκες διαφορετικές μεν, πλησίον δε των «Α», με ίδιους ή περίπου  ίδιους στόχους, με περισσότερους ή λιγότερους συναφείς περιορισμούς. Ο τρόπος επιλογής του συνόλου αυτών των «σχετικών» σχεδιασμών, οι οποίοι πλέον θα αποκαλούνται «σχεδιασμοί βάσης»,  δεν εμπίπτει στα ενδιαφέροντα της διατριβής. Επίσης, δεν εμπίπτει η διαδικασία κοινής παραμετροποίησής τους, για την περίπτωση που αυτή δεν υφίσταται, μιας και αυτό μπορεί να πραγματοποιηθεί με πολλούς τρόπους. Η διατριβή προτείνει ένα νέο μαθηματικό τρόπο έκφρασης κάθε νέου σχεδιασμού συναρτήσει των σχεδιασμών βάσης. Ο τρόπος αυτός παρακάμπτει ουσιαστικά την παραμετροποίηση της σχεδιαζόμενης γεωμετρίας, η οποία εκ των πραγμάτων μπορεί να εισάγει εκατοντάδες βαθμούς ελευθερίας, καθυστερώντας έτσι τη σύγκλιση της βασισμένης σε ΕΑ μεθόδου βελτιστοποίησης. Αντ’ αυτών, εισάγει ένα πραγματικά μικρό πλήθος νέων μεταβλητών σχεδιασμού αλλά και τον καθορισμό περιοχών  μεγαλύτερης σημαντικότητας, με όφελος τη σημαντική μείωση του χρόνου βελτιστοποίησης. Επιπλέον σημαντικό κέρδος από τη νέα παραμετροποίηση αποτελεί το ότι τα όρια των νέων μεταβλητών σχεδιασμού προκύπτουν εύκολα, ουσιαστικά «αυτόματα» και χωρίς παρέμβαση του χρήστη. Η προτεινόμενη μέθοδος προγραμματίστηκε συμπληρωματικά στον \english{EASY} και χρησιμοποιήθηκε για το σχεδιασμό θερμικών και υδροδυναμικών μηχανών με μιας τάξης μεγέθους κέρδος σε υπολογιστικό χρόνο, για την επίτευξη  παρόμοιας ποιότητας σχεδιασμών.       
               
	β)  Προτείνεται μέθοδος εντοπισμού και, κυρίως, χρήσης των όποιων συσχετίσεων μεταξύ των μεταβλητών σχεδιασμού για την περαιτέρω επιτάχυνση των κλασικών ΕΑ, επεμβαίνοντας στους τελεστές εξέλιξης  Αυτό επιτυγχάνεται μέσω της δυναμικά (δηλαδή σε κάθε γενιά) ανανεούμενης επαναδιατύπωσης του προβλήματος βελτιστοποίησης ώστε να χειρίζεται κατά το δυνατό μη-συσχετιζόμενες (διαχωρίσιμες) μεταβλητές σχεδιασμού. Ο λόγος που μια τέτοια αντιμετώπιση επιφέρει κέρδος στο χρόνο βελτιστοποίησης είναι ότι οι ΕΑ, εκ φύσεως, κερδίζουν σημαντικά σε ταχύτητα όταν χειρίζονται προβλήματα διαχωρίσιμων μεταβλητών σχεδιασμού. Υπενθυμίζεται ότι κάθε πρόβλημα ελαχιστοποίησης μιας συνάρτησης Ν διαχωρίσιμων μεταβλητών μπορεί να αντιμετωπιστεί ως Ν προβλήματα ελαχιστοποίησης μιας συνάρτησης μιας μεταβλητής, με συνολικά μικρότερο υπολογιστικό κόστος.  Αυτό εκμεταλλεύεται η προτεινόμενη μέθοδος. Προαπαίτηση για την υλοποίηση μιας τέτοιας μεθόδου είναι η αυτόματη μετατροπή του πραγματικού προβλήματος σε πρόβλημα διαχωρίσιμων μεταβλητών. Αυτό πραγματοποιείται με χρήση της μεθόδου ανάλυσης σε κύριες συνιστώσες (ΑσΚΣ, \english{Principal Component Analysis, PCA}). Σε προβλήματα πολυκριτηριακής βελτιστοποίησης, η ΑσΚΣ εφαρμόζεται στο σύνολο των επιλέκτων κάθε γενιάς και, ουσιαστικά, πραγματοποιεί κατάλληλη «στροφή» του χώρου σχεδιασμού. Η «στροφή» αυτή απαιτεί την επίλυση ενός προβλήματος ιδιοτιμών. Αποδεικνύεται ότι ο προκύπτων χώρος, ίδιας διάστασης με το χώρο σχεδιασμού, ο οποίος καθορίζεται με άξονες τα ιδιοδιανύσματα (τις λεγόμενες «κύριες συνιστώσες») που προκύπτουν από την διαδικασία ΑσΚΣ, έχει τις κατά το δυνατό μη-συσχετιζόμενες μεταβλητές σχεδιασμού. Οι τελεστές εξέλιξης (διασταύρωση, μετάλλαξη, κλπ) εφαρμόζονται στις νέες μεταβλητές σχεδιασμού, προκύπτουν οι νέοι απόγονοι, οι οποίοι τελικά επαναφέρονται (με «αντίθετη στροφή») στον αρχικό-πραγματικό χώρο σχεδιασμού. Η προτεινόμενη μέθοδος προγραμματίστηκε συμπληρωματικά στον \english{EASY} και αρχικά πιστοποιήθηκε σε μαθηματικές συναρτήσεις και ψευδο-μηχανολογικά προβλήματα της βιβλιογραφίας. Στη συνέχεια, χρησιμοποιήθηκε για το σχεδιασμό του δρομέα ενός υδροστροβίλου  \english{HYDROMATRIX\circledR} με κέρδος την αναπαραγωγή παρόμοιας ποιότητας σχεδιασμών στο μισό χρόνο (μισός αριθμός κλήσεων του λογισμικού αξιολόγησης) σε σχέση με τον κλασικό EA.  
 
	γ) Ως προς τον ΕΑ που υποβοηθείται από μεταπρότυπα (\english{MAEA}, στη λογική της προσεγγιστικής προ-αξιολόγησης – ΠΠΑ – των ατόμων κάθε γενιάς με τεχνητά νευρωνικά δίκτυα) προτείνεται και πιστοποιείται μέθοδος η οποία αντιμετωπίζει με επιτυχία ένα σημαντικό πρόβλημα τέτοιων μεθόδων όταν χρησιμοποιούνται σε προβλήματα μεγάλης διάστασης. Είναι γνωστό από τη μέχρι τώρα εμπειρία από τη χρήση \english{MAEA} ότι το κέρδος (συγκριτικά με τους κλασικούς ΕΑ) μειώνεται όταν η διάσταση του χώρου σχεδιασμού αυξάνει σημαντικά. Αυτή η συμπεριφορά οφείλεται (α) στο ότι η έναρξη χρήσης των τεχνητών νευρωνικών δικτύων καθυστερεί, αναμένοντας την καταγραφή επαρκών ήδη αξιολογημένων υποψηφίων λύσεων στη βάση δεδομένων από την οποία αντλούνται τα δείγματα εκπαίδευσης των μεταπροτύπων και (β) στο ότι η αξιοπιστία των νευρωνικών δικτύων φθίνει αυξάνοντας τον αριθμό εισόδων σε αυτά (πλήθος μεταβλητών σχεδιασμού). Η προτεινόμενη αντιμετώπιση αυτού του προβλήματος βασίζεται στην ελεγχόμενη μείωση των εισόδων (κρατώντας, ουσιαστικά, τις περισσότερο αντιπροσωπευτικές) του νευρωνικού δικτύου (εδώ, δικτύου ακτινικής βάσης, \english{radial basis function network, RBF}) που χρησιμοποιείται  ως μεταπρότυπο.  Χρησιμοποιώντας και πάλι ανάλυση σε κύριες συνιστώσες, στο ίδιο σύνολο δυναμικά ανανεούμενων επιλέκτων, πραγματοποιείται εκ νέου στροφή/ευθυγράμμιση του χώρου σχεδιασμού με τις κύριες συνιστώσες λαμβάνοντας υπόψη την σημαντικότητα της κάθε μεταβλητής. Η τελευταία είναι αντιστρόφως ανάλογη της τιμής της σχετικής ιδιοτιμής που προέκυψε από την ΑσΚΣ  που έγινε κατά τη διάρκεια της μεθόδου. Το επιπλέον στοιχείο, εδώ, είναι ότι στο «στραμμένο» πλέον χώρο σχεδιασμού, γίνεται αποκοπή και συγκρατείται μικρός αριθμός των πλέον σημαντικών «στραμμένων» μεταβλητών σχεδιασμού. Με τις τελευταίες, και μόνο αυτές, εκπαιδεύεται το δίκτυο \english{RBF}. Η τεχνική αυτή οδηγεί σε περαιτέρω μείωση του χρόνου βελτιστοποίησης αφού τα μεταπρότυπα παρέχουν προβλέψεις υψηλότερης αξιοπιστίας αλλά και μπορούν να  ξεκινήσουν να χρησιμοποιούνται νωρίτερα κατά τη διαδικασία βελτιστοποίησης. Η μέθοδος χρησιμοποιήθηκε, με ή χωρίς την επικουρική χρήση της μεθόδου (α) και σε συνδυασμό με τη μέθοδο (β) στο σχεδιασμό-βελτιστοποίηση 2Δ και 3Δ πτερύγωσης συμπιεστή με κέρδος τη μείωση του χρόνου βελτιστοποίησης στο 1/3 του χρόνου ενός κλασικού \english{MAEA}.        

Οι τεχνικές που αναπτύχθηκαν στην παρούσα διδακτορική εργασία χρησιμοποιήθηκαν στο σχεδιασμό και βελτιστοποίηση ενός καινοτόμου τύπου υδροστροβίλου “\english{HYDROMATRIX\circledR}” ιδανικού για τοποθεσίες μικρού ύψους (\english{H=5-10m, Q$>$60m3/s}). Οι υδροστρόβιλοι “\english{HYDROMATRIX\circledR}”, πατενταρισμένοι από την εταιρία \english{Andritz-Hydro} συνδυάζουν τη υψηλή αποδοτικότητα (10\% υψηλότερη από άλλους τύπους υδροστροβίλων χαμηλού ύψους) με το χαμηλό κόστος εγκατάστασης, με περιβαλλοντικά και οικονομικά οφέλη. Κάνοντας χρήση των τεχνικών που αναπτύχθηκαν στην παρούσα διδακτορική διατριβή, επιτεύχθηκε μείωση του συνολικού χρόνου σχεδιασμού (όχι μόνο της διαδικασίας βελτιστοποίησης) σε ποσοστό μεγαλύτερο του 50\% από 120-140 μέρες σε μόλις 50.  Η προαναφερθείσα μείωση του συνολικού χρόνου σχεδιασμού και η συνεπαγόμενη μείωση του κόστους σχεδιασμού μετατρέπουν σε οικονομικά επικερδή τη χρήση “\english{HYDROMATRIX\circledR}” σε ακόμη μικρότερα ύψη, \english{H=2-5m}, και παροχές μικρότερες των \english{60m3/s}. Αξίζει να σημειωθεί ότι η διαθεσιμότητα τοποθεσιών με αυτά τα χαρακτηριστικά, μόνο στην Ευρώπη, είναι της τάξης των \english{6000 GWh/year} με ότι αυτό συνεπάγεται τόσο σε οικονομικά οφέλη όσο και σε περιβαλλοντολογικά λόγω της αντίστοιχης μείωσης των εκπομπών \english{CO2}. 

Στο πλαίσιο συνεργασιών με την εταιρία \english{Andritz-Hydro}, η οποία χρηματοδότησε τμήμα της  παρούσας διατριβής, οι αναπτυχθείσες τεχνικές εφαρμόστηκαν και σε μια σειρά άλλων σχεδιασμών υδροδυναμικών μηχανών ή συνιστωσών τους. Περισσότερο αναλυτικά, με βάση τις τεχνικές αυτές οργανώθηκαν διαδικασίες σχεδιασμού δρομέων όλων των κλασικών τύπων υδροδυναμικών στροβιλομηχανών αντίδρασης τόσο αξονικής όσο και μικτής ροής (\english{Francis, Kaplan, Bulb, Pump} και \english{Pump-Turbines}), με στόχο τόσο την αύξηση της απόδοσής τους όσο και τη βέλτιστη συνεργασία τους με τα υπόλοιπά μέρη της εγκατάστασης. Επίσης σχεδιάστηκαν σταθερές συνιστώσες υδροδυναμικών μηχανών όπως αγωγοί απαγωγής (\english{draft tubes}), με στόχο τη μέγιστη ανάκτηση πίεσης με τις ελάχιστες απώλειες και τμήματα αγωγών εισόδου υδροστροβίλων δράσης διανομείς-\english{distributors}), με στόχο την ελαχιστοποίηση του κόστους κατασκευής και απωλειών με παράλληλη αύξηση της ποιότητας της δέσμης ρευστού μετά το ακροφύσιο. Επιλεγμένο τμήμα των παραπάνω σχεδιασμών παρουσιάζεται στο κείμενο της διδακτορικής διατριβής. 

Το έργο με τίτλο «\english{HYDROACTION – Development and laboratory testing of improved action and Matrix hydro turbines designed by advanced analysis and optimization tools» (Project Number 211983)}, το οποίο χρηματοδότησε η Ευρωπαϊκή Ένωση, υποστήριξε το υπόλοιπο τμήμα της διατριβής. Το ΕΜΠ και η \english{Andritz-Hydro} υπήρξαν εταίροι στο έργο αυτό. Οι βιομηχανικής κλίμακας υπολογισμοί, στις προαναφερθείσες εφαρμογές θερμικών και υδροδυναμικών στροβιλομηχανών, πραγματοποιήθηκαν, εκ των πραγμάτων, σε πολυεπεξεργαστικά συστήματα. Ως τέτοια χρησιμοποιήθηκαν τα πολυεπεξεργαστικά συστήματα της ΜΠΥΡ\&Β/ΕΘΣ (Αθήνα) και της \english{Andritz-Hydro (Linz, Graz \& Vevey)}, ενίοτε και συνεργατικά μέσω τεχνικών \english{Grid Computing}, τις οποίες υποστηρίζει το λογισμικό \english{EASY}. 
%\end{abstracts}

\end{abstractslongGR}

% ---------------------------------------------------------------------- 
