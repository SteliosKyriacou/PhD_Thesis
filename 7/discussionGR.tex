% this file is called up by thesis.tex
% content in this file will be fed into the main document

\chapter{Σύνοψη - Συμπεράσματα - Παρόν \& Μέλλον} % top level followed by section, subsection


% ----------------------- paths to graphics ------------------------

% change according to folder and file names
\ifpdf
    \graphicspath{{7/figures/PNG/}{7/figures/PDF/}{7/figures/}}
\else
    \graphicspath{{7/figures/EPS/}{7/figures/}}
\fi

% ----------------------- contents from here ------------------------
Αντικείμενο της διδακτορικής διατριβής ήταν να εμπλουτίσει και να επεκτείνει υπάρχουσες μεθόδους (και λογισμικό) βελτιστοποίησης βασισμένο στους ΕΑ που έχει αναπτυχθεί στη ΜΠΥΡ\&Β/ΕΘΣ του ΕΜΠ κατά την τελευταία δεκαετία. Κύριος στόχος της ήταν να επιτευχθεί ικανοποιητική μείωση του χρόνου αναμονής του σχεδιαστή κατά την επίλυση βιομηχανικών εφαρμογών μεγάλης κλίμακας. Οι προτεινόμενες μέθοδοι και το προγραμματισθέν λογισμικό χρησιμοποιήθηκαν σε αρκετές εφαρμογές σχεδιασμού-βελτιστοποίησης στις στροβιλομηχανές, θερμικές και υδροδυναμικές. Αρκετές πιστοποιήσεις και αξιολογήσεις μεθόδων έγιναν σε γρήγορα προβλήματα ελαχιστοποίησης μαθηματικών συναρτήσεων ή σε 2Δ εφαρμογές βελτιστοποίησης στις θερμικές στροβιλομηχανές. Όμως, οι κύριες εφαρμογές έγιναν σε προβλήματα βιομηχανικού ενδιαφέροντος, στην ευρύτερη περιοχή των στροβιλομηχανών. 

Ως προς τα στοιχεία πρωτοτυπίας της διατριβής, σχετικά με την ανάπτυξη νέων μεθόδων και τον προγραμματισμό νέου λογισμικού, η συνεισφορά της διατριβής έγκειται στα παρακάτω :
\begin{itemize}
\item[]{\bf α)} στη δημιουργία μεθόδου σχεδιασμού στη βάση αρχειοθετημένης γνώσης (\english{Knowledge-Based Design} ή \english{KBD}).
\item[]{\bf β)} στην ανάπτυξη και χρήση εξελικτικών τελεστών υποβοηθούμενων από \newline ΑσΚΣ κατά τη βελτιστοποίηση μέσω ΕΑ και
\item[]{\bf γ)} στην ανάπτυξη και χρήση υποβοηθούμενων από ΑσΚΣ μεταπροτύπων κατά τη βελτιστοποίηση μέσω ΜΑΕΑ, δηλαδή ΕΑ που χρησιμοποιούν τεχνητά νευρωνικά δίκτυα ως τοπικά μεταπρότυπα.
\end{itemize}

Κάνοντας χρήση των προτεινόμενων στη διατριβή μεθόδων επιτυγχάνεται α) η βέλτιστη εκμετάλλευση διαθέσιμων παλαιότερων επιτυχημένων σχεδιασμών, οδηγώντας σε αναδιατύπωση του προβλήματος βελτιστοποίησης με πολύ λιγότερες μεταβλητές και, άρα, σημαντική μείωση του χρόνου επίλυσης των προβλημάτων μέσω ΕΑ ή ΜΑΕΑ. β) Η διατήρηση των πλεονεκτημάτων των προϋπαρχουσών μεθόδων επιτάχυνσης των ΕΑ (κυρίως της, βασισμένης στα μεταπρότυπα, προσεγγιστικής προ-αξιολόγησης του πληθυσμού κάθε γενιάς) σε προβλήματα μεγάλης διάστασης (50-500 μεταβλητές σχεδιασμού), και γ) η περαιτέρω επιτάχυνση των ΕΑ από την αυτόματη και συνεχώς ανανεούμενη αναδιατύπωση και επίλυση των προβλημάτων βελτιστοποίησης σε νέα μορφή με, κατά το δυνατό, διαχωρίσιμη συνάρτηση κόστους. Κατέστη έτσι δυνατή η χρήση μεθόδων βελτιστοποίησης βασιζόμενες στους ΕΑ σε προβλήματα βιομηχανικής κλίμακας (προβλήματα, δηλαδή, με μεγάλη διάσταση και  δύσκολες μη-διαχωρίσιμες συναρτήσεις-στόχους), επιστρέφοντας σχεδιασμούς υψηλής ποιότητας σε μειωμένο χρόνο σχεδιασμού. Ενδεικτικά, κατά το σχεδιασμό υδροστροβίλου  επιτεύχθηκε μείωση του συνολικού χρόνου σχεδιασμού (όχι μόνο της διαδικασίας βελτιστοποίησης) άνω του 50\%, από 120 ως 140 μέρες σε μόλις 50.  Η προαναφερθείσα μείωση του συνολικού χρόνου σχεδιασμού και η συνεπαγόμενη μείωση του κόστους σχεδιασμού μετατρέπουν σε οικονομικά επικερδή τη χρήση του σε ακόμη μικρότερα ύψη, \english{H=2-5m}, και παροχές μικρότερες των $60m^3/s$.

Το λογισμικό που αναπτύχθηκε στην παρούσα διατριβή χρησιμοποιείται, ήδη, στη βιομηχανία και συγκεκριμένα σε μια από της μεγαλύτερες εταιρίες στον τομέα των υδροηλεκτρικών έργων, την \english{Andritz-Hydro}. Στην εταιρεία αυτή, γίνεται χρήση του λογισμικού, σε τακτική φάση, στο σχεδιασμό δρομέων υδροδυναμικών μηχανών τύπου α)\english{Francis}, β)\english{Kaplan}, γ)\english{Bulb},  δ)\english{Pump} και ε)\english{Pump-Turbine} αλλά και στατικών μερών υδροδυναμικών μηχανών όπως αγωγών απαγωγής και προσαγωγής. Αυτή τη στιγμή, υπάρχει ικανός αριθμός εγκατεστημένων ή προς εγκατάσταση σχεδιασμών ανά τον κόσμο που πραγματοποιήθηκαν κάνοντας χρήση των μεθόδων και του λογισμικού της παρούσας διατριβής.  

\section{Μελλοντική Ερευνα}
Ακολουθούν προτάσεις για μελλοντική έρευνα, που μπορούν να αποτελέσουν φυσική συνέχεια της διατριβής αυτής: 
\begin{itemize}

\item{}Διερεύνηση της πιθανότητας ύπαρξης και εκμετάλλευσης μη-γραμμικών συσχετίσεων των μεταβλητών σχεδιασμού (σε σχέση με τη συνάρτηση-στόχο του προβλήματος), σε «κακώς-τοποθετημένα» προβλήματα βελτιστοποίησης.

\item{}Χρήση κατανεμημένου ΕΑ, βασισμένου στον επιμερισμό του συνόλου των επιλέκτων με τρόπο ώστε ο κάθε δήμος (ή υποπληθυσμός, στην ορολογία των ΕΑ) να αναλαμβάνει μία περιοχή του μετώπου \english{Pareto} και να διαμορφώνει το δικό του διαχωρίσιμο πρόβλημα, βασισμένος στους «δικούς του> επιλέκτους.

\item{}Διερεύνηση τρόπων αντιμετώπισης προβλημάτων βελτιστοποίησης με μεγάλη θνησιμότητα μέσω της χρήσης διαφορετικού τύπου εξελικτικών τελεστών κατά τα πρώτα στάδια της εξέλιξης. Αυτό θα ήταν ιδιαίτερα χρήσιμο σε δύσκολα προβλήματα βελτιστοποίησης με πολλούς αυστηρούς περιορισμούς, όπου η θνησιμότητα των αναδυομένων λύσεων οφείλεται, ακριβώς, στη μη-ικανοποίηση των περιορισμών. 	

\item{}Υβριδισμός ΕΑ με αιτιοκρατικές μεθόδους σε βιομηχανικά προβλήματα, εφαρμόζοντας μεθόδους πολυεπίπεδης βελτιστοποίησης. Αυτό απαιτεί τη χρήση συζυγών τεχνικών (\english{adjoint methods}, για τον υπολογισμό το παραγώγων των μετρικών ποιότητας ή συνδυασμών αυτών) και αναπτύσσοντας τρόπους μορφοποίησης πλέγματος (\english{mesh morphing}) σύμφωνα με τις υπολογισθείσες παραγώγους. Αντίστοιχος υβριδισμός έχει πραγματοποιηθεί σε άλλα προβλήματα σε παλαιότερες διατριβές στη ΜΠΥΡ\&Β/ΕΘΣ του ΕΜΠ.% Σε προβλήματα υδροδυναμικών μηχανών, αντίστοιχα με αυτά που εξετάζονται σε αυτήν τη διατριβή, βρίσκεται υπό εξέλιξη η ανάπτυξη των απαραίτητων συζυγών μεθόδων και αυτές θα παρουσιαστούν στη διδακτορική διατριβή του Ε. Παπουτσή-Κιαχαγιά.    

%\item{} Χρήση μεταπροτύπων κατά την πολυεπίπεδη αξιολόγηση για να εντοπισθούν οι διαφορές, όσον αφορά στις μετρικές ποιότητας, μεταξύ των προτύπων αξιολόγησης διαφορετική πιστότητας και κατάλληλη αλλαγή των στόχων για το επίπεδο που κάνει χρήση του προτύπου χαμηλής πιστότητας.       

\item{} Ενδιαφέρον  παρουσιάζει, επίσης, η χρήση ΕΑ με μεταβαλλόμενο μέγεθος γενιάς ούτως ώστε να κρατούνται σταθερά μεγέθη όπως η πολυμορφικότητα (αποφυγή πρόωρης ή λανθασμένης σύγκλισης του ΕΑ) ή το υπολογιστικό κόστος ανά γενιά (όταν λ.χ. ένα άτομο της παρούσας γενιάς υπάρχει ήδη στη βάση δεδομένων τότε να δίνεται η δυνατότητα στον ΕΑ να υπολογίζει κάποιο άλλο στη θέση του) κλπ.       


\end{itemize}



\section{∆ημοσιεύσεις από τη διατριβή}
Παρατίθενται οι δημοσιεύσεις σε επιστημονικά περιοδικά και οι παρουσιάσεις σε επιστημονικά συνέδρια που πραγματοποιήθηκαν κατά την εκπόνηση της διατριβής.
\english{
\newline 
\newline 
S. Erne, M. Lenarcic, \underline{S.A. Kyriacou}.
Shape Optimization of a Flows Around Circular Diffuser in a Turbulent Incompressible Flow. ECCOMAS 2012 Congress,
Vienna, Austria, September 10-14 2012.
\newline 
\newline 
K.C.Giannakoglou, V.G. Asouti, \underline{S.A. Kyriacou}, X.S. Trompoukis: ‘Hierarchical, Metamodel–Assisted Evolutionary Algorithms, with
Industrial Applications’, von Karman Institute Lectures Series on ‘Introduction to Optimization and Multidisciplinary Design in
Aeronautics and Turbomachinery’, May 7-11, 2012.
\newline 
\newline 
\newline 
\newline 
\underline{S.A. Kyriacou}, S. Weissenberger and K.C. Giannakoglou.
Design of a Matrix hydraulic turbine using a metamodel-assisted evolutionary algorithm with PCA-driven evolution
operators. International Journal of Mathematical Modelling and Numerical Optimization, SI: Simulation-Based
Optimization Techniques for Computationally Expensive Engineering Design Problems, 3(2):45–63, 2012.
\newline 
\newline 
I.A. Skouteropoulou, \underline{S.A. Kyriacou}, V.G Asouti, K.C. Giannakoglou, S. Weissenberger, P. Grafenberger.
Design of a Hydromatrix turbine runner using an asynchronous algorithm on a multi-processor platform. 7th GRACM
International Congress on Computational Mechanics, Athens, 30 June-2 July, 2011.
\newline 
\newline 
\underline{S.A. Kyriacou}, S. Weissenberger, P. Grafenberger, K.C. Giannakoglou.
Optimization of hydraulic machinery by exploiting previous successful designs. 25th IAHR Symposium on Hydraulic
Machinery and Systems, Timisoara, Romania, September 20-24, 2010.
\newline 
\newline 
H.A. Georgopoulou, \underline{S.A. Kyriacou}, K.C. Giannakoglou, P. Grafenberger and E. Parkinson.
Constrained multi-objective design optimization of hydraulic components using a hierarchical metamodel-assisted
evolutionary algorithm. Part 1: Theory. 24th IAHR Symposium on Hydraulic Machinery and Systems, Foz do Iguassu,
Brazil, October 27-31, 2008.
\newline 
\newline 
P. Grafenberger, E. Parkinson, H.A. Georgopoulou, \underline{S.A. Kyriacou} and K.C. Giannakoglou.
Constrained multi-objective design optimization of hydraulic components using a hierarchical metamodel-assisted
evolutionary algorithm. Part 2: Applications. 24th IAHR Symposium on Hydraulic Machinery and Systems, Foz do
Iguassu, Brazil, October 27-31, 2008.
}


% ---------------------------------------------------------------------------
% ----------------------- end of thesis sub-document ------------------------
% ---------------------------------------------------------------------------